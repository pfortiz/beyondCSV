\ifdefined\inglese
\addcontentsline{toc}{chapter}{Introduction}
\chapter*[Introduction]{Introduction}

%\chapter{Before and After the Internet\label{book-FFbeforeandafter}}

%\headfigure{images/book-FFbeforeandafter.jpg}

\subliminal{The way we were}

\begin{comment}
+webDescription
Generic content
-webDescription
\end{comment}

\fi


\ifdefined\espanol
\addcontentsline{toc}{chapter}{Introducci\'on}
\chapter*[Introducci\'on]{Introducci\'on}

\subliminal{No todo lo que brilla es oro}


La \internet\ ha sido la revoluci\'on tecnol\'ogica que
m\'as r\'apido se ha propagado en la historia de la humanidad. 
En menos de 30 a\~nos, esta tecnolog\'ia pas\'o de tener 
poco menos de un mill\'on de usuarios
a estar disponible para cuatro mil millones de personas en todo el mundo, y
pas\'o de estar disponible para un grupo de adultos j\'ovenes, a estar
disponible para todas las edades: de 0 a m\'as de 100 a\~nos.

La \internet\ no se trata de una sola tecnolog\'ia ni de un solo objetivo,
y no habr\'ia podido nacer sin estar asociado al desarrollo de computadores
y microelectr\'onica. La \internet\ principalmente se trata de
transmitir informaci\'on en forma digital remotamente y en tiempo real;
pero transmitir no ser\'ia suficiente para explicar lo que hoy vivimos,
tambi\'en se requiri\'o de la capacidad de almacenar informaci\'on en forma
digital, de organizar dicha informaci\'on, de saber c\'omo encontrar
la informaci\'on que nos interesa en un cierto momento y de tener acceso a
ella lo m\'as sencillamente posible para que la \internet\ nos fuera
\'util.
Hoy no podamos ni siquiera imaginar nuestras vidas sin todas las
herramientas que se han desarrollado a su alrededor.

\myIdea{Rese\~na personal}

Los autores de este libro, Patricio y Sebastian somos padre e hijo.
Patricio naci\'o d\'ias antes que Yuri Gagarin se convirtiera en el primer
hombre en conquistar el espacio; en ese a\~no, NASA contaba s\'olo con un
computador electr\'onico, que funcionaba a medias, y en ese entonces, la
palabra computador se asociaba con una persona con muchas habilidades en
matem\'aticas capaz de realizar complejas operaciones en un corto tiempo.
Sebastian naci\'o un a\~no despu\'es del desastre de Chernobyl y ya la
\internet\ exist\'ia, aunque estaba disponible a una fracci\'on peque\~na
de la poblaci\'on: personas estudiando carreras cient\'ificas en
universidades del primer mundo (como fue el caso de Patricio), personas
trabajando para grandes instituciones o empresas, o personas envueltas en
proyectos asociados a gobiernos del grupo G7, y por supuesto, \oxm a nadie se le
hubiera ocurrido que un ni\~no pudiera tener acceso a esta tecnolog\'ia;
solo el
costo de un PC era alt\'isimo y no eran juguetes\cxm

Quienes crearon la \internet\ lo hicieron en un ambiente para adultos,
nunca pensaron que los ni\~nos podr\'ian tener acceso a ella y por lo tanto
no construyeron salvaguardas para protegerlos. En esos a\~nos, las
salvaguardas ven\'ian dadas por el acceso restringido a instituciones, el
alto costo de los equipos y al hecho que si bien prender un computador lo
pod\'ia hacer cualquiera, lograr hacer que este hiciera algo \'util
requer\'ia de bastante conocimiento espec\'ifico.

\oqm Qui\'en hubiera imaginado hace 5 a\~nos que ya entrando al segundo
a\~no de la pandemia del \coronavirus, estar\'iamos trabajando desde la
casa, y nuestros hijos estudiando remotamente (desde casa),
apoy\'andonos en la \internet, las tecnolog\'ias que la posibilitan y
aquellas incentivadas por su existencia\cqm

\myIdea{General description, pros and cons}

Como toda tecnolog\'ia, la \internet\ tiene una serie de virtudes que
son innegables, y que han cambiado nuestras vidas para bien, facilitado
cosas que antes pod\'iamos tardar semanas o meses en conseguir y
posibilitado cosas que simplemente no exist\'ian. Por otro lado, tambi\'en
como toda tecnolog\'ia, tiene cosas que \odq peligrosas\cdq\ para sus
todos sus usuarios, sean adultos, j\'ovenes o ni\~nos.

Un m\'etodo que muchos estudios utilizan para comprender los efectos de
un comportamiento o una actividad define dos aspectos: los \bf{efectos
primarios}, que se refieren a los efectos que experimenta la persona que
se comporta de una forma o que realiza una actividad, y los \bf{efectos
secundarios}, que se refieren a c\'omo el que alguien realice una
actividad repercute en su entorno cercano (familia, amistades) y
tambi\'en en la sociedad. Este m\'etodo lo aplicaremos a varias
situaciones.


\myIdea{Tecnolog\'ia activa, no pasiva y orientada comercialmente}

La \web\ y la \internet\ es tecnolog\'ia de comunicaci\'on activa, pues
todos podemos utilizarla, todos podemos leer material que encontremos y
sobre todo, todos podemos publicar material en ella. Cuando la \internet\
se comercializ\'o, otras fuerzas comenzaron a operar. Fuerzas que le
ayudaron a expandirse, pero fuerzas que ven en cada uno de nosotros un
blanco para entregarnos informaci\'on lo m\'as pr\'oxima a nuestros gustos,
y en ello, las redes sociales han jugado un rol fundamental. Podemos
encontrar lo que se nos antoje, y se nos sugiere ver o informarnos de cosas
asociadas a nuestros gustos, pero tambi\'en se nos ofrecen \odq cosas\cdq\
que est\'an \odq algo m\'as all\'a\cdq del c\'irculo de cosas que nos
gustan, pudiendo llevarnos a lugares que no planeamos o en el caso de los
ni\~nos, a lugares decididamente peligrosos, inadecuados, o que van contra
nuestra visi\'on como padres.

\myIdea{Un ejercicio de risk assesment}

\myIdea{Descripci\'on por cap\'itulo}

El hecho que existen peligros nos insta a replantearnos en el primer
cap\'itulo preguntas b\'asicas
como \oqm Por qu\'e protegemos a nuestros ni\~nos\cqm, \oqm C\'omo
protegemos a nuestros ni\~nos\cqm, y \oqm De qu\'e protegemos a nuestros
ni\~nos\cqm,  aunque la \'ultima est\'a tratada en mayor profundidad en el
cap\'itulo \sc{Exposici\'on y uso de la Web}

Para entender los beneficios y los riesgos de la \web, el Cap\'itulo
\it{\oqm Es buena o mala la web\cqm} enfoca el tema partiendo con una
pregunta que no se puede responder en forma tajante; es por eso que 
dividimos el problema en dos partes: el aspecto tecnol\'ogico y lo que
se refiere al contenido que est\'a disponible en la \web\ y otras
plataformas.

La \web\ es s\'olo una de las componentes de la interacci\'on ni\~nos-\web.
La otra son los ni\~nos y c\'omo se desarrollan y evolucionan; este tema lo
tratamos en el cap\'itulo \sc{El desarrollo de nuestros hijos}.

Todos tenemos h\'abitos, algunos tenemos adicciones a algo o tenemos
tendencia a \odq enviciarnos\cdq\ con algo, como ver a cada
rato lo que nos ofrece \instagram, o tocar frecuentemente el \sphone\ para
ver si tenemos mensajes. En \sc{h\'abitos, adicciones y
publicidad} profundizamos en el tema.

\oqm Qui\'en podr\'ia dudar de que la \web, la \internet\ y otras tecnolog\'ias han
tenido un gran impacto en la educaci\'on\cqm\ Hablamos de impacto positivo
y negativa y en \sc{Impacto en la educaci\'on} reflexionamos recogemos la
opini\'on de expertos en el tema.

Como padres tenemos la responsabilidad de guiar lo mejor posible a nuestros
hijos, asegurando que su formaci\'on avance al paso adecuado,
tomando medidas para que en la \web\ tengan a acceso a sitios seguros (o lo
m\'as seguros posible). \sc{Responsabilidad de los padres} cubre esos
temas.

La \web\ no ser\'ia la \web\ ni tendr\'ia el contenido que hoy tiene ni
albergar\'ia los comentarios e interacciones que vemos sin la componente
humana. En \sc{espejo de la humanidad} describimos lo que hemos observado
en los \'ultimos 30 a\~nos, c\'omo actuamos cuando pensamos que otros no
nos ven y c\'omo somos en l\'inea. Lo hemos hecho para que el lector piense
e asocie actitudes con personas con las que le haya tocado interactuar.

Finalmente, la \web\ est\'a aqu\'i para quedarse, los gobiernos puede que
se vean obligados a legislar para hacerla m\'as segura y amigable,
entonces, para terminar, nos preguntamos \sc{Qu\'e hacemos entonces}

% (y abuelos y t\'ios) 
% hace un peque\~no recorrido por sus luces y sus sombras as\'i
% como tambi\'en un ana\'alisis de los aspectos tecnol\'ogicos asociados a  
% la \internet. 

%% <El comercio electr\'onico; el seguimiento de env\'ios en tiempo real,
%% <los tr\'amites en l\'iea; el correo electr\'onico, que permite
%% <comunicarnos de manera casi instant\'anea con una o m\'as personas a
%% <miles de kil\'ometros de distancia; la adquisici\'on de conocimientos
%% <que antes s\'olo pod\'ian hacerse mediante libros; as\'i como la
%% <oportunidad de informarnos, o incluso de entretenernos sin depender de
%% <la programaci\'on que nos ofrecen emisoras de radio y canales de
%% <televisi\'on, son s\'olo algunas de las virtudes de esta tecnolog\'ia
%% <que cambi\'o muchos h\'abitos que ten\'iamos.

%-- Sin embargo, este crecimiento acelerado y en ocasiones completamente
%-- descontrolado, ha significado desaf\'ios gigantescos para todos quienes
%-- formamos parte de esta \bf{Red Global de Informaci\'on} que se sustenta
%-- en la \internet.  S\'i, aumque parezca extra\~no, somos parte activa
%-- de esta red global de informaci\'on, ya que sin saberlo, alimentamos
%-- esta red con nuestros datos: ubicaci\'on, preferencias de b\'usqueda,
%-- compras hechas, e incluso conversaciones que pudi\'eramos haber tenido.
%-- Todo esto queda guardado en alg\'un servidor, sin que siquiera sepamos.
%-- 
%-- 
%-- A pesar de lo que nos trate de vender la industria, estas tecnolog\'ias
%-- a\'un no maduran, 
%-- por lo que es necesario realizar estudios serios 
%-- sobre los efectos de su uso, 
%-- as\'i como del contenido que por ellas se propaga.
%-- Tan r\'apido ha sido este desarrollo, que
%-- reci\'en en los \'ultimos 8 a\~nos se est\'an escuchando voces que
%-- nos advierten de los peligros a que estamos expuestos, pero sobre todo
%-- los peligros a los que est\'an expuestos nuestros hijos, \oxm Y s\'i que hay peligros en el mundo digital\cxm\
%-- tantos o m\'as que en el ``mundo real'',
%-- por lo que es nuestro deber como padres protegerlos de la mejor manera posible, tal
%-- como lo hacemos en los dem\'as aspectos de la vida.
%-- 
%-- Dado el grado de penetraci\'on de estas tecnolog\'ias en nuestra vida cotidiana, 
%-- y como interacctuan con nuestro cerebro 
%-- (incluida nuestra percepci\'on del mundo), cualquier estudio
%-- es necesariamente de largo aliento, tal como fueron los estudios sobre
%-- los efectos del tabaco en la salud humana, en ese caso, un estudio de
%-- un a\~no no hubiera sido suficiente para detectar los efectos a largo alcance;
%-- con la \internet\ estamos en una situaci\'on similar.
%-- 
%-- Cuando hablamos de identificar peligos no s\'olo hablamos de conocimiento
%-- o de normas escritas, sino tambi\'en de experiencias vividas por nosotros 
%-- o aprendidas de otras personas, principalmente mayores. 
%-- La \internet\ es tan nueva que
%-- hay poco conocimiento, pocas normas, y
%--  la primera generaci\'on que naci\'o
%-- y creci\'o con ella a\'un no es adulta, por lo que el traspaso de experiencias 
%-- a una segunda generaci\'on ocurrir\'a en no menos de una d\'ecada.. 
%-- 
%-- %En cuanto al conocimiento sobre sus potenciales peligros, 
%-- %reci\'en en la \'ultima d\'ecada han comenzado a salir los primeros estudios
%-- %sobre los posibles efectos adversos de usar esta tecnolog\'ia de forma prolongada,
%-- %algunos hablando de efectos f\'isicos, sobre todo cuando hablamos de ni\~nos peque\~nos,
%-- %y otros refiri\'endose a los efectos psicol\'ogicos de la exposici\'on prolongada
%-- %a contenidos no apropiados para ni\~nos o adolescentes.
%-- 
%-- %ESTA BUENO PARA LA PARTE DE LAS LEYES
%-- 
%-- %En cuanto a las normas, sin duda ser\'an las que m\'as lento avances,
%-- %ya que por cada norma que se plantee, aparece\'a 
%-- %un grupo de inter\'es que se opondr\'a a su implementaci\'on, 
%-- %y los grupos interesados en que no hayan mayores
%-- %restricciones son incre\'iblemente poderosos.
%-- 
%-- La identificaci\'on de peligros lleva a la busqueda de protecci\'on
%-- contra \'estos, por lo que en el contexto del desarrollo de nuestros
%-- hijos se hace necesario explicitar a a qu\'e nos referimos con
%-- protecci\'on digital. Nos referimos a las distintas formas de proteger a
%-- nuestros hijos en el Cap\'itulo I: Protegemos a nuestros hijos.
%-- 
%-- %Cuando hablamos de protecci\'on nos referimos a todas las peque\~nas cosas 
%-- %que como padres hacemos para resguardar la seguridad de nuestros hijos.
%--  
%-- % Cuando reci\'en nacidos lo \'unico de lo que nos preocupamos 
%-- % es de protegerlos: darles leche materna para desarrollar el apego,
%-- % protegerlos de los cambios de temperatura, entre muchas otras cosas;
%-- % no les damos comida s\'olida, porque sabemos que sus organismos 
%-- % podr\'ian procesarla, por lo que esperamos a que tengan las condiciones para ello.
%-- % 
%-- % Ya cuando comienzan a caminar y son m\'as aut\'onomos,
%-- % nos esmeramos en que no est\'en a su alcance objetos peligrosos como cuchillo y otros objetos 
%-- % filoss o que podr\'ian quebrarse. No comenzamos a bajar estas cosas 
%-- % ni descuidarnos hasta que les hemos ense\~nado que estos objetos son peligrosos.
%-- % Mientras tanto los vamos vacunando, para evitar que
%-- % se enfermen.
%-- % 
%-- % Ya m\'as grandes, y por tanto a\'un m\'as aut\'onomos, 
%-- % vamos dej\'andolos poco a poco salir solos, 
%-- % primero cerca de la casa \odq donde nuestros ojos los vean\cdq
%-- % para luego ir perdi\'endolos de vista a medida que van creciendo.
%-- % C\'omo padres  seguramente recordamos alguna de las primeras veces en 
%-- % que los dejamos salir solos y se nos perdieron, o la primera vez que 
%-- % salieron solos con amigos, cuando les d\'abamos una hora de regreso.
%-- % 
%-- % Ahora bien, no s\'olo nosotros nos hemos dado a la tarea de
%-- % cuidar a nuestros hijos, tambi\'en lo hace el estado al implementar ciertas 
%-- % restricciones. Horarios para mayores en la televisi\'on, sillas para ni\~nos en los autos,
%-- % prohibici\'on de la venta de alcohol y tabaco a menores de edad,
%-- % la prohibici\'on de mantener relaciones sexuales con menores de edad, 
%-- % son muchas de las restricciones que ha consensuado la sociedad
%-- % a fin de proteger de una u otra forma a nuestros ni\~nos.
%-- 
%-- Pero \oqm Qu\'e sucede con la \internet\cqm 
%-- 
%-- La \internet, al igual que las otras tecnolog\'ias de punta no fue
%-- dise\~nada para ser usadas por ni\~nos. Hoy se recomienda que menores de
%-- edad lo tengan acceso a \internet\ bajo supervisi\'on de sus padres u otros adultos.
%-- No se comenzaron a imprimir libros para ni\~nos sino hasta siglos
%-- despu\'es que Gutemberg invent\'o la imprenta. A\'un no hay autos para que
%-- los conduzcan ni\~nos por la calle; hay ni\~nos que manejan autos, pero son
%-- especiales y en ambientes altamente controlados --muchos de los \'ultimos
%-- campeones de \it{F\'ormula 1} han conducido por m\'as de 20 a\~nos cuando alcanzan los 30
%-- a\~nos. 
%-- 
%-- Lamentablemente en la \web, salvo contadas excepciones, no existen 
%-- estos \odq ambientes controlados\cdq\ y seguros. Una raz\'on es que no hay
%-- mecanismos de reconocimiento de edad; y la otra es que hay
%-- grandes intereses econ\'omicos interesados en tener acceso al mercado que
%-- representan los ni\~nos.
%-- 
%-- % otra raz\'on es porque los intereses en que no 
%-- %los haya son muy grandes y poderosos: quien env\'ia nuestros correos,
%-- %la tienda virtual donde compramos, o la red social en la que
%-- %publicamos buena parte de nuestra vida no est\'an interesados
%-- %en que nuestros hijos, que tambi\'en son proveedores de datos
%-- %queden fuera de este gigantesco mercado.
%-- 
%-- % C\'omo ya mencionamos, los avances en legislaci\'on 
%-- % ocurren a ritmo muy lento, llegando varios a\~nos m\'as tarde 
%-- % de lo que era necesario. En ese sentido debemos ser 
%-- % conscientes que nuestros hijos no son un laboratorio en el que
%-- % podamos experimentar varias veces:
%-- % aquello que le demos o no les quedar\'a para toda su vida,
%-- % es por esto que es de vital importancia tomar los resguardos 
%-- % necesarios para exponerlos lo menos posible a los peligros del 
%-- % mundo digital de igual manera como lo hacemos en el mundo real. 
%-- % 
%-- % La sociedad ha consensuado que el trabajo infantil,
%-- % o que golpear a los ni\~nos no es sano para ellos, sin embargo
%-- % \oqm Sabemos qu\'e cosas no debemos permitir que hagan en la Web\cqm
%-- % 
%-- % Si bien podemos tener algunas nociones, 
%-- % como evitar que vean pornograf\'ia o violencia extrema, 
%-- % es bueno hilar m\'as fino en cuanto a la protecci\'on que queremos entregar
%-- % \oqm Conozco al detalle cu\'ales son las amenazas que hay en la \web \cqm
%-- % \oqm Conozco alg\'un mecanismo para atenuar estas amenazas\cqm
%-- % 
%-- % Ambas preguntas no son triviales, y son las que nos han motivado
%-- % a escribir este libro, porque cuando hablamos de los efectos que 
%-- % puede provocar la \web\ en nosotros o nuestros hijos, hablamos
%-- % de una multitud de disciplinas que convergen en ese aparato llamado
%-- % computador o celular. Neurociencia, psicolog\'ia, an\'alisis de datos y de patrones,
%-- % marketing entre muchas otras.
%-- 
%-- 
%-- Tambi\'en se hace necesario recordar como nos vamos desarrollando como
%-- seres humanos, incluyendo el desarrollo neuronal y ps\'iquico 
%-- de ni\~nos y adolescentes, para entender c\'omo se desarrolla su 
%-- cerebro, as\'i como su personalidad. Estos temas son cubiertos en el
%-- Cap\'itulo \it{El desarrollo de nuestros ni\~nos}
%-- 
%-- En \it{H\'abitos, adicciones y publicidad} hablamos sobre como nuestro comportamiento es
%-- completamente maleable y puede ser modificado,
%-- en particular por quienes nos rodean, as\'i como por la publicidad,
%-- que sin que lo notemos, va moldeando nuestras preferencias.
%-- La psicolog\'ia juega un rol important\'isimo en ello.
%-- Por su importancia, tambi\'en hablaremos de la publicidad, 
%-- de como incide en los ni\~nos, y de lo relevante que es este mercado.
%-- 
%-- Preguntas importantes para los padres de la era digital
%-- son: \oqm A qu\'e darles acceso y a qu\'e edad\cqm\ 
%-- \oqm A qu\'e se exponen nuestros hijos al navegar s\'olos en la \internet \cqm\
%-- \oqm Existen espacios especiales para ni\~nos en la \web \cqm\
%-- y por \'ultimo \oqm C\'omo puedo proteger a mis hijos del contenido no apropiado para ellos\cqm\
%-- Nos referiremos a estas preguntas en el cap\'itulo \it{Responsabilidad de los padres}. 
%-- 
%-- Ya conscientes de la existencia de peligros en la \internet, en el cap\'itulo
%-- \it{Exposici\'on y uso de la Web} abordaremos los temas m\'as delicados
%-- a los que ni\~nos y adolescentes pueden verse expuestos: 
%-- ciberacoso en redes sociales, la pedofilia, y el gran n\'umero de 
%-- aparatos por los que se puede acceder a la \web.
%-- 
%-- La \internet\ ha cambiado a la sociedad a todo nivel, 
%-- incluyendo la educaci\'on. 
%-- La facilidad de obtener informaci\'on 
%-- ha contribuido a tener m\'as fuentes de las cuales estudiar,
%-- aunque tambi\'en ha repercutido en una disminuci\'on
%-- en la capacidad de memorizar, ello debido a que hoy
%-- en d\'ia casi toda la informaci\'on est\'a a unos pocos \it{clicks} de distancia.
%-- De esto hablaremos en detalle en el cap\'itulo \it{Impacto en la educaci\'on}.
%-- 
%-- En el \'ultimo cap\'itulo \it{\oqm Qu\'e hacemos entonces\cqm}\
%-- elaboramos, a partir de todo lo antes visto, 
%-- una serie de consejos para los padres, 
%-- y tocamos el tema de la necesidad imperiosa de que los gobiernos legislen
%-- para proteger a los ni\~nos en la \web.
%-- 
%-- %Para cerrar, hablaremos de la experiencia internacional en el control 
%-- %de la \internet, esta experiencia a\'un es muy poca a nivel mundial,
%-- %por eso es necesario que como padres tomemos consciencia 
%-- %de lo importante que es el cuidado de nuestros hijos,
%-- %en un mundo que, producto del \bf{Coronavirus} se ha hecho a\'un m\'as virtual.
%-- 
%-- Este libro originalmente formaba parte de un libro m\'as grande,
%-- sin embargo, los autores hemos considerado que 
%-- producto del encierro a causa del \bf{Coronavirus}, 
%-- y del incremento en el tiempo de permanencia de 
%-- nuestros hijos frente a una pantalla se hace imperativo 
%-- educar a los padres acerca de los riesgos que hay en la \web,
%-- y a\'un m\'as importante, ense\~narles a tomar los resguardos
%-- necesarios para que sus hijos queden lo menos expuestos posible.
%-- 
%-- Es importante se\~nalar que este libro no cuenta con financiamiento
%-- de ninguna empresa de \internet, 
%-- de lo contrario, no hubi\'esemos sido capaces de decir ni un cuarto de 
%-- lo que hemos dicho.
%-- 

\myIdea{Estilo del libro}

Estando muy conscientes de la importancia de no hacer distinciones de
g\'enero, sobre todo en lo que se refiere a roles asociados
tradicionalmente a ni\~nos y ni\~nas, en el libro utilizaremos las
herramientas ya existentes en el espa\~nol para as\'i promover una lectura
m\'as flu\'ida. En espa\~nol \bf{ni\~nos} puede significar \it{un conjunto de
hombres menores de edad} \bf{o} \it{un conjunto de hombres y mujeres menores de
edad}.

Decir \odq Creemos que es peligroso para los ni\~nos\cdq\ se lee mucho
m\'as fluido que \odq Creemos que es peligroso para los ni\~nos y las
ni\~nas\cdq, o peor, \odq Creemos que es peligroso para l@s ni\~n@s\cdq\
o incluso \odq Creemos que es peligroso para lus ni\~nus\cdq.  Cuando
usamos la palabra \odq ni\~nos\cdq, el lector deber\'a comprender que
nos referimos a \odq ni\~nos y ni\~nas\cdq, y si el lector decide pasar
informaci\'on a sus hijos (= hijos o hijas), queda en libertad de
expandir el significado en el modo que le agrade.

Tambi\'en usamos mucho la expresi\'on \bf{los padres}, y con ella nos
referimos cualquiera de las combinaciones que se dan hoy en d\'ia, y
haya o no un certificado de matrimonio de por medio, o se trate o no de
padres biol\'ogicos.

Nosotros no discriminamos, eso debe quedar claro desde el principio.

\myIdea{Closure}

Queremos agradecer a la Dra. Valeria Rojas Osorio (vice presidenta de la
sociedad de neurolog\'ia infantil de Chile) por sus valiosos comentarios
del cap\'itulo dedicado al desarrollo de los ni\~nos. A Barbara Ortiz por
hacer una acucioso prueba de lectura, y a muchos otros que nos dieron su
opini\'on o contribuyeron en aspectos como mejorar la escritura o con
ejemplos valiosos.

Patricio y Sebastian, en medio de la cuarentena del \it{novel coronavirus}
\fi
